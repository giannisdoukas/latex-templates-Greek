% !TEX encoding = UTF-8 Unicode
% !TEX TS-program = XeLaTeX

\documentclass[a4paper,11pt]{article}

%gia glwssa
\usepackage{fontspec}
\setmainfont{Times New Roman}
\setsansfont{Arial}
%\newfontfamily\greekfont[Script=Greek]{Times New Roman}
%\newfontfamily\greekfontsf[Script=Greek]{Times New Roman}
\usepackage{polyglossia}
\setdefaultlanguage{greek}
%telos gia glwssa

% load package with some of the available options 
\usepackage[framed,numbered,autolinebreaks,useliterate]{mcode}

%gia tis listes
\usepackage{enumerate}

%gia tis eikones
\usepackage{graphicx}
\DeclareGraphicsExtensions{.pdf,.png,.jpg, .eps}

%gia ta mathimatika kai symbola
\usepackage{mathtools}
\usepackage{amssymb}

%o counter gia ta equation na exei pio prin to section
\renewcommand{\theequation}{\thesubsection .\arabic{equation}}

%gia tis grammes sto panw kai sto katw meros tis sellidas
\usepackage{fancyhdr}
\usepackage{lastpage}
\pagestyle{fancy}
\fancyhf{}
\fancyhead[LE,RO] {\LaTeX}
\fancyhead[RE,LO]{Template}
\fancyfoot[CE,CO]{Page \thepage\ of \pageref{LastPage}}

\renewcommand{\headrulewidth}{2pt}
\renewcommand{\footrulewidth}{1pt}

%gia na exoun link ta periexomena
\usepackage{color}   %May be necessary if you want to color links
\usepackage{hyperref}
\hypersetup{
    colorlinks=false, %set true if you want colored links
    linktoc=all,     %set to all if you want both sections and subsections linked
    linkcolor=black,  %choose some color if you want links to stand out
}

%gia ti lista sti vivliografia
\renewcommand{\theenumi}{[\arabic{enumi}] }

%gia ta link
\usepackage{hyperref}

%command to plot with label
\renewcommand{\thefigure}{\thesection .\arabic{figure} }

\usepackage{framed}
\newcommand{\showplot}[2]{
\begin{figure}[htbp]
	\centering
	\includegraphics[scale=0.55]{#1} 
	\caption{#2}
\end{figure}
}


\newcommand{\norm}[1]{\left\lVert#1\right\rVert} %norm symbol

\begin{document}

\begin{titlepage}
	\newcommand{\HRule}{\rule{\linewidth}{0.5mm}} % Defines a new command for the horizontal lines, change thickness here
	\center
	
	\textsc{\LARGE Τιτλος 1}\\[1.5cm] % Name of your university/college
	\textsc{\Large Τιτλος 2}\\[0.5cm] % Major heading such as course name
	\textsc{\large Τιτλος 3}\\[10ex] % Minor heading such as course title

	\HRule \\[0.4cm]
	{ \huge \bfseries Τιτλος}\\[0.4cm] % Title of your document
	\HRule \\[1.5cm]
 	
	\textsc{Created with \LaTeX}
	
	\vfill 
	\begin{minipage}{0.4\textwidth}
	\begin{flushleft} \large
	Ονομα \textsc{Επιθετο} 
	\end{flushleft}
	\end{minipage}
	~
	\begin{minipage}{0.4\textwidth}
	\begin{flushright} \large
	email@mail.com
	\end{flushright}
	\end{minipage}\\[4cm]
	\pagebreak
	
\end{titlepage}

\clearpage{\thispagestyle{empty}}\mbox{}\clearpage 
\tableofcontents{ \thispagestyle{empty}}
\newpage


\renewcommand\thesection{}
\section{Εισαγωγή}
	Τα χαρακτηριστικά του υπολογιστικού συστήματος στο οποίο υλοποιήθηκαν οι ασκήσεις στο περιβάλλον matlab περιγράφονται στον παρακάτω πίνακα. \\

	\begin{table}[h]
	\centering
		\begin{tabular}{| c | c | l l l l l l l l l}
		\hline
		Model Name: 						& 	MacBook Pro 9.2									\\
		L2 Cache (per Core):				&	256 KB											\\
		L3 Cache:							&	3 MB												\\
		Merory (Ram):						&	4 GB												\\
		Operating System:					&	OS X 10.9.5										\\
		Matlab Version:						&	8.3.0.532 (R2014a) 64-bit 								\\
		Διακριτότητα Χρονομετρητή:			&	2.9254e-07										\\
		Αποτέλεσμα LU από εντολή bench:	&	0.1141											\\
		mex -setup						&	Xcode with Clang									\\
		\hline
		\end{tabular}
	\end{table}

	
\pagebreak
\renewcommand\thesection{\arabic{section}}
\setcounter{section}{0}	%μιδενίζεται ο counter του section

\section{Κεφάλαιο} %kefalaio 2
	Εδώ ξεκινάει ένα καινούριο κεφάλαιο.
	\subsection{Υποκεφάλαιο}			
				
		Ο πίνακας \ref{tab:label} είναι στη σελίδα \pageref{tab:label}.
		\begin{table}[htbp]
  			\centering
  			\begin{tabular}{@{} c | ccc @{}}
    				\hline
    				Εμπρός Σχετικό σφάλμα					& 		dgesv 		&	dgesvx		& 	mldivide 		\\ 
    				\hline
    				τυχαίο με δείκτη κατάστασης $10^4$ 		& 	 	1.7546e-12	&	2.7660e-12	&  	1.2803e-12	\\ 
    				τυχαίο με δείκτη κατάστασης $10^{10} $	& 		4.2675e-12	&	6.9389e-13	& 	2.8894e-12	\\ 
    				τυχαίο κάτω τριγωνικό 					& 		3.2660e-11	&	1.2412e-13	& 	1.0223e-11	\\ 
    				gfpp 									&		1.1708e-09	&	8.5931e-14	&	1.1775e-09	\\ 
    				\hline
  			\end{tabular}
  			\caption{Αποτελέσματα}
  			\label{tab:label}
		\end{table}
		
		
	% βιβλιογραφία
	\clearpage
	\renewcommand\thesection{}
	\pagebreak
	\section{Βιβλιογραφία}	
	
	\begin{enumerate}
		\item Gilbert Strang. Εισαγωγή στη γραμμική άλγεβρα. Εκδόσεις Πανεπιστήμιο Πατρών 2012
		\item \href{http://www.mathworks.com/help/matlab/}{\textit{http://www.mathworks.com/help/matlab/}}
		\item Scientific Computing:An Introductory Survery. Michael T. Heath
	\end{enumerate}
	 	
\end{document}

		
